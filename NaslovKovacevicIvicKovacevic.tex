% !TEX ecnoding = UTF-8 Unicode

\documentclass[a4paper]{article}

\usepackage{color}
\usepackage{url}
\usepackage[T1]{fontenc}
\usepackage[utf8]{inputenc}
\usepackage{graphicx}

\usepackage{listings}

\usepackage[english,serbian]{babel}

\usepackage[unicode]{hyperref}
\hypersetup{colorlinks, citecolor=green,filecolor=green,linkcolor=blue,urlcolor=blue}

\begin{document}

\title{Naslov\\ \small{Seminarski rad u okviru kursa\\Metodologija stručnog i naučnog rada\\ Matematički fakultet}}

\author{Aleksandra Kovačević, David Ivić, Sreten Kovačević\\ aleksandrakovacevic099@gmail.com, 310ddavi@gmail.com,\\ sretenkvcvc94@gmail.com}
\date{26.~mart 2015.}
\maketitle

\abstract{
}

\tableofcontents

\newpage

\section{Uvod}
\section{Bezbednost veb servera}


Sa pojavom Veb 2.0, pove\'{c}anjem deljenja informacija preko socijalnih mre\v{z}a i kori\v{s}\'{c}enjem veba kao sredstva poslovanja i pru\v{z}anja razli\v{c}itih informacija bezbednost sajtova \v{c}esto biva ugro\v{z}ena direktnim upadima od strane zlonamernih programera.

Zlonamerni programeri(u daljem tekstu hakeri) ili nastoje da kompromituju korporativnu mre\v{z}u ili navode krajnje korisnike koji pristupaju mre\v{z}i ka preuzimanju sadr\v{z}aja \v{c}ijeg rizika nisu svesni (virusi). Kao rezultat toga, industrija obra\'{c}a veliku pa\v{z}nju na bezbednost samih veb aplikacija, pored bezbednosti osnovne ra\v{c}unarske mre\v{z}e i operativnih sistema.

\textbf{Bezbednost veb sajtova} je grana bezbednosti informacija koja se bavi bezbedno\v{s}\'{c}u veb sajtova, veb aplikacija i veb servera. Bezbednost veb aplikacija se oslanja na principe bezbednosti aplikacija uop\v{s}te, ali ih primenjuje specifi\v{c}no za internet i veb sisteme.
Bezbednost veb servera je za\v{s}tita informacija dostupnih preko veb servera.


\subsection{Sigurosni rizici}
Najve\'{c}i sigurnosni rizici poti\v{c}u iz nerazumevanja sposobnosti hakera. Kada se razmi\v{s}lja o za\v{s}titi aplikacije, ne sme se voditi mi\v{s}lju: sve \v{s}to je prikazano, biva na\v{s}e ograni\v{c}enje onoga \v{s}to mo\v{z}emo da u\v{c}inimo. Sve \v{s}to predstavlja klijentsku stranu i ograni\v{c}enja koja su na njoj nametnuta se veoma lako mogu zaobi\'{c}i.
\textbf{Pretra\v{z}iva\v{c} ne predstavlja ograni\v{c}enje }.\\
?U okrviru ovog rada, razmatra\'{c}emo tri velika sigurnosna rizika:
\begin{enumerate}
	\item Cross-Site Scripting (XSS)
	\item SQL Injection
	\item DOPUNA
\end{enumerate}
\newpage
\subsection{Cross-Site Scripting (XSS)}
\textbf{Cross Site Scripting, poznatiji kao XSS} je napad injekcijom kodova u sajt. Takodje je jedan od najuobi\v{c}ajnih na\v{c}ina napada, jer svaki sajt zahteva od korisnikovog pretra\v{z}iva\v{c}a podr\v{s}ku javascript-a.\\
\textbf{Problem:} Korisnik je u situaciji da po\v{s}alje neki vid odgovora stranici(searchbox), a ako taj odgovor sadr\v{z}i neke HTML tagove ili javascript kod, stranica ih renderuje i izvr\v{s}ava kod.\\
\textbf{Posledice:} \begin{enumerate}
	\item kradja ID-a sesije i sasim tim kradja korisnikovog indentitea(u cilju obavljanja poslova u ime tog korisnika)
	\item kontrola stranice (blokiranje rada, menjanje pojedina\v{c}nih elemenata...)
	\item redirekcija na drugu, zlonamernu stranicu
\end{enumerate}
\subsubsection{XSS primeri}
\textit{Reflexted XSS}\\\\
Reflected XSS se javlja kada se napada zasniva na injektovanju izvr\v{s}nog koda u HTTP odgovor. Kod nije sa\v{c}uvan u okviru same aplikacije, ve\'{c}(u naivnim primerima) mo\v{z}e biti vidljiv u okviru nadogradjenog url-a(kao posledica http odgovora).\\
Razmotrimo slede\'{c}i primer:\\
Predpostavimo da imamo jednostavan sajt, \v{c}ija je uloga da ima ima jedan searchbox
\begin{enumerate}
\item nakon bilo kakve pretrage, promeni\'{c}e nam se url (www.sajt.com?key="pretrazujemo")
\item izmenom sadr\v{z}aja sekcije 'key', mo\v{z}emo da ubacimo neki skript
\begin{lstlisting}
<script>alert("123")</script>
\end{lstlisting}
To isto mo\v{z}emo posti\'{c}i i sa uno\v{s}enjem takvog koda unutar searchbox-a
\item Uno\v{s}enjem takvog url-a u na\v{s} pretra\v{z}iva\v{c} izlazi\'{c}e nam poruka "123"

\end{enumerate}
Naravno, ovo je jednostavan primer bez vidljivih znakova opasnosti, ali na sli\v{c}nom principu deluju zlonamerni mail-ovi koji nam pristi\v{z}u ili sajtovi(ili \v{c}ak korisnici) koji name\'{c}u linkove da korisnik klikne na njih.\\
Na\v{c}in prikrivanja ovakvih napada(jer korisnik mo\v{z}e da nasluti iz url-a da je re\v{c} o linku ka zlonamernoj strani) jesu URL shortener aplikacije.\\\\
\textit{Stored XSS}\\\\
Suprotno Reflected XSS-u koji ne ostavlja tragove na samoj aplikaciji, odnosno unatar njenog koda je Stored XSS. Cilj stored XSS-a je da sakrije svoju zlonamernost tako \v{s}to je javan. Najprostiji primer jeste da u okviru napadnute aplikacije, haker ostavi link ka zlonamernoj stranici ili skidanje istog takvog programa.
\subsubsection{XSS prevencija}
Podsetimo, da je glavni problem upravo to \v{s}to pretra\v{z}iva\v{c} renderuje HTML tagove i izvr\v{s}ava javascript kod  
\section{Zaključak}

\addcontentsline{toc}{section}{Literatura}
\appendix
\bibliography{seminarski}
\bibliographystyle{plain}




\end{document}